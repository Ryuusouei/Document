\documentclass[12pt,a4paper,fleqn]{book}	%設定文件內容字體大小:12pt,文件格式:report(另有article(期刊)、book(書)等格式)
%	fleqn => 使數學式靠左對齊,預設是居中對齊。
%	A4 paper 為 21cm*29.7cm


\usepackage{booktabs}   		%設計表格內容

\usepackage{xeCJK}       		%打中文必備,會自動載入fontspec,並讓讓中英文字體分開設置

\renewcommand{\figurename}{圖}
\renewcommand{\tablename}{表}

\setCJKmainfont{標楷體}				%設定預設中文字型:標楷體
%以下為字體切換的設定,設定方法:內容文字前依照格式輸入{\Hei XXXX},{\MS XXXX}及可變成相對應字體
\newCJKfontfamily\Hei{微軟正黑體}  	%內容文字前依照此格式輸入{\Hei XXXX},輸出文字可改成:微軟正黑體		
\newCJKfontfamily\NewMing{新細明體}	%內容文字前依照此格式輸入{\NewMing XXXX},輸出文字可改成:新細明體
\newCJKfontfamily\MS{MS Mincho}		%內容文字前依照此格式輸入{\MS XXXX},輸出文字可改成:MS PGothic

\setmonofont[Scale=0.8]{Courier New}	% 等寬字型

\XeTeXlinebreaklocale "zh"			%指示Latex以中文的方式斷行
\XeTeXlinebreakskip = 0pt plus 1pt	%在字元間加入0pt~1pt的彈性間距

\title{CAN controller with transceiver (CS8972)}
\author{Jing-Jou Tang, \\
Tsung-Ying Liu, \\
STUST TJJ Group}

\date{\today} %不要日期

\graphicspath{{figure/}{table/}}	%設定圖片及圖片(表格)的搜索路徑

\begin{document}


\maketitle

    \setcounter{page}{2}			%頁碼開始計算的編號
	\pagenumbering{roman}			%用羅馬rs頁碼

%%% 建立目次、表目錄與圖目 %%%
    \tableofcontents
    \addcontentsline{toc}{chapter}{\hspace{-4.5ex}目次}
    \listoftables
    \addcontentsline{toc}{chapter}{\hspace{-4.5ex}表目錄}
    \listoffigures
    \addcontentsline{toc}{chapter}{\hspace{-4.5ex}圖目錄}
    \newpage

    \setcounter{page}{2}			%頁碼開始計算的編號
	\pagenumbering{arabic}			%用一般數字編號


  	\chapter{CAN Introduction}
	隨著科技不斷進步,人們對於生活中的各種需求不斷提高,而汽車工業也不斷演進,......
	\section{What's CAN}
	講解CAN的歷史及由來
		\subsection{Features}
		說明CAN的特性及優缺點
		\subsection{Protocol}
		介紹CAN的通訊協定
	\section{CAN controller and CAN transceiver}
	淺談 CAN controller and CAN transceiver 的歷史,簡單帶到
		\subsection{Introduction}
		詳細說明並介紹 CAN Controller 及 CAN Transceiver 的發展
		\subsection{Description}
		說明 CAN Controller 及 CAN Transceiver 的功能與作用,並帶到
		\subsection{CENTURY MYSON's CAN Controller}
			Single to talk about CAN controller CS8972 及帶入下一章		%載入content
	\chapter{CS8972's Introduction}
簡單說明MYSON 的CS8972這顆IC的由來
	\section{Description}
	說明IC的功能及應用範圍
	
	\section{Features}
	說明其特性

	\section{Registers}
	說明所有CS8972的暫存器			%載入content
	\chapter{Control CS8972 using {$SPI$} interface}
簡單說明CS8972 與SPI 有關的部份
	\section{Protocol}
	說明SPI的規範

	\section{Features}
	說明SPI特性及功能

	\section{Registers}
	說明CS8972與SPI相關的暫存器
	
	\section{Example (with MEGAXXXX)}
	說明CS8972與MEGAXXXX透過SPI介面的溝通時的操作,及範利例解說					%載入content
	\chapter{Control CS8972 using {$I^2C$} interface}
簡單說明CS8972 與IIC 有關的部份
	\section{Protocol}
	說明IIC的規範

	\section{Features}
	說明IIC特性及功能

	\section{Registers}
	說明CS8972與IIC相關的暫存器
	
	\section{Example (with MEGAXXXX)}
	說明CS8972與MEGAXXXX透過IIC介面的溝通時的操作,及範利例解說					%載入content
	\chapter{Application}
說明可利用CS8972來實際應用的一些系統,及最後可用不同的CAN control and CAN transceiver來溝通的一些實例
	\section{How to transmit by CS8972s}
	說明如何利用 CS8972 來做傳送的範例
			
	\section{How to receive by CS8972}
	說明如何利用 CS8972 來做接收的範例

	\section{System 1}
	說明一些系統應用的實例1
	
	\section{System 2}
	說明一些系統應用的實例1	
	
	\section{Different CAN Bus hardware configuration}
		MEGAWINXXXX+CS8972 and MEGAWINXXXX+MCP2551+MCP2515
		
	\section{Socket CAN}			%載入content




	\newpage
    \addcontentsline{toc}{chapter}{\hspace{-4.5ex}Reference}	%將Reference的頁數也加入到目錄當中

	\bibliographystyle{IEEEtran}								%Reference依照IEEEtran格式
	\bibliography{test2OPC_bib_20120821}							%載入.bib檔

\end{document}
