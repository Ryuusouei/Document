	\begin{frame}[label=title]%---------title page 
    	\titlepage\end{frame} 


	\begin{frame}%---------outline page  
		\frametitle{Outline}
		\begin{columns}
			\begin{column}{6cm}		
				\begin{itemize}
					\item \hyperlink{introduction}{What's the XeLaTeX?}
					\item \hyperlink{install}{Installation}
					\item \hyperlink{first}{Create my first one tex file.}
				\end{itemize}
			\end{column}
			
			\begin{column}{6cm}
				\includegraphics[scale=0.15]{FirstTexFile}
			\end{column}
		\end{columns}	
	\end{frame}
		

	\begin{frame}[label=introduction]%---------introduction page
		\frametitle{What's the TeX?}
		Tex is a typesetting system.
		\begin{columns}
			\begin{column}{7cm}		
				\begin{itemize}
					\item {Initial release \alert{1978} (35 years ago) }
					\item {Developer is \alert{Donald Knuth}}
					\item {Operating system	\alert{Cross-platform}}
					\item {\alert{Free} software}
				\end{itemize}
			\end{column}
			\begin{column}{4cm}
				\TeX\newline
				\LaTeX\newline
				\LaTeXe\newline
				\XeTeX\newline
				\XeLaTeX
			\end{column}
		\end{columns}
	\end{frame}

	\begin{frame}[label=install]%---------installation page
		\frametitle{Installation}
		This document is for Windows System, if you want to install \XeLaTeX, here are the steps I took:\newline		 
		\begin{enumerate}
			\item {loaded TeX Live, \newline
				\alert{Official Website: }\url{http://www.tug.org/texlive/}} 
			\item {loaded Texmaker, \newline
				\alert{Official Website: }\url{http://www.xm1math.net/texmaker/}}
			\item {modify configuration of Texmaker}
		\end{enumerate}
	\end{frame}

	\begin{frame}[fragile]%---------create first one tex page
		\frametitle{To modify configuration of Texmaker}
		Start Texmaker\newline
		Manu-Options -> Configure Texmaker
		\newline
		\begin{center}
		\includegraphics[scale=0.3]{ConfigureTexmaker}
		\end{center}
		
	\end{frame}
	
	\begin{frame}[label=TexmakerHotKey]%---------introduction page  
		\frametitle{Hot Keys of Texmaker}
		\begin{columns}
			\begin{column}{6cm}		
				\includegraphics[scale=0.4]{TexmakerHotKey}
			\end{column}
			\begin{column}{4cm}
				\begin{tabular}{cc}
					Function & Hot Key\\
					\hline
					\\
					LaTeX &  F2 \\
					\\
					瀏覽 PDF & F7 \\
					\\					
					BibTeX & F11 \\
				\end{tabular}			
			\end{column}
		\end{columns}
	\end{frame}	
		
	\begin{frame}[fragile]%---------create first one tex page
		\frametitle{Create a tex file(1/2)}
		\begin{lstlisting}
			%========================================
			\documentclass{article}
			\usepackage{xeCJK}

			\setmainfont{Times New Roman}
			\setCJKmainfont{標楷體}

			\begin{document}
			     Hello everybody. お早う皆さん! 大家好!
			\end{document}
			%========================================
		\end{lstlisting}
		This txt save as a *.tex file, and then enter F2 -> F7, finally you will get the *.pdf file in the same path.
	\end{frame}

	\begin{frame}%---------create first one tex page
		\frametitle{Create my first one tex file(2/2)}
		
		\centering
		\includegraphics[scale=0.3]{HelloEverybody}

	\end{frame}

	\begin{frame}[label=Education]%---------reference page
		\frametitle{education}
		\begin{thebibliography}{10}
			\bibitem{LaTeX/Mathematics}
			{\em LaTeX/Mathematics}, \url{http:/http://en.wikibooks.org/wiki/LaTeX/Mathematics}, 2013.
			\bibitem{LaTeX/Tables}
			{\em LaTeX/Tables}, \url{http://en.wikibooks.org/wiki/LaTeX/Tables}, 2013.
			\bibitem{LaTeX/Floats, Figures and Captions}
			{\em LaTeX/Floats, Figures and Captions}, \url{http://en.wikibooks.org/wiki/LaTeX/Floats,_Figures_and_Captions}, 2013.
			\bibitem{Tex}
			{\em Tex}, \url{http://tex.stackexchange.com/}, 2013.
		\end{thebibliography}
	\end{frame}

	\begin{frame}[label=reference]%---------reference page
		\frametitle{Reference}
		\begin{thebibliography}{10}
			\bibitem{TeX Live}
			{\em TeX Live}, \url{http://www.tug.org/texlive/}, 2013.
			\bibitem{Texmake}
			{\em Texmaker}, \url{http://www.xm1math.net/texmaker/}, 2013.
			\bibitem{用Bemaer做簡報}
			蔡炎龍, {\em 用Bemaer做簡報}, 2010.
		\end{thebibliography}
	\end{frame}

	\begin{frame}{完}
		\frametitle{Finial page...}
		\centering
		\Large{Thank you for your attemation!}
	\end{frame} 

	\begin{frame}[label=o]%---------Video page
		\includegraphics[scale=0.1]{FirstTexFile}

		\end{frame}
		\begin{frame}[label=Video]%---------Video page
		\frametitle{Video testing}
		\centering
		\href{movie/123.mp4}{\includegraphics[scale=0.5]{pcaSpiral}}

	\end{frame}	
	
	\begin{frame}[label=Picture]%---------Picture page
		\frametitle{Picture testing}
		\centering
		\pgfputat{\pgfxy(3,-1)}{\includegraphics[scale=0.3]{pcaSpiral}}

	\end{frame}	


	\begin{frame}	
		\unitlength=1mm
		\begin{picture}(80, 60)
			\multiput(5, 0)(5, 0){15}{\line(0, 1){60}} % 畫 15 條直線,每隔 5mm 一條
			\multiput(0, 5)(0, 5){11}{\line(1, 0){80}} % 畫 11 條橫線,每隔 5mm 一條
			\thicklines
			\put(0, 0){\vector(0, 1){60}} % 畫 y 軸
			\put(0, 0){\vector(1, 0){80}} % 畫 x 軸
			\put(0, 0){\circle*{1}}       % 畫圓點,實心粗點
			\put(-5, -5){$O(0, 0)$}       % 標上原點的座標
			\put(-5, 60){$y$}             % 標上 y 軸字樣
			\put(80, -5){$x$}             % 標上 x 軸字樣

			\put(-7, 20){\rotatebox{90}{這裡是 $y$ $axis$}}
			\put(30, -6){這裡是 $x$ $axis$}

		\end{picture}
	\end{frame}


	\begin{frame}	
		This is picture example. See Fig. ~\ref{fig:test} 
		\begin{figure}[ht]	%[h] figure put on the end of this passage 
							%[t] figure put on the top of the page
							%[b] figure put on the bottom of the page 
			\centering
			\includegraphics[scale=0.4]{pcaSpiral}
			\caption{ テスト}
			\label{fig:test}

		\end{figure}
	\end{frame}


	\begin{frame}
		\begin{tabbing}
	
			xxxxxxxxxx\=xxxxxxxxxx\=xxxxxxxxxx\kill
			column1	\> column2	\> column3	\\
			item1	\> item2	\> item3	\\
			itema	\> itemb	\> itenc

		\end{tabbing}

		\begin{table}[h]
			\caption{TestTable}
			\centering
			\begin{tabular}[t]{lll} %{lll} {ccc} {rrr} = left center right
				\hline
					& column2 & column3	\\
				\cline{2-3}
				item1	& item2	& item3	\\
				itemA	& itemB	& itemC	\\
				\hline
			\end{tabular}
		\end{table}


		\begin{table}[h]
    		\centering
			\begin{tabular}{lcc}
			\hline
			義大利 & 0.5  & 16.12 \\\hline
			英國   & 2    & 12.3 \\\hline
			加拿大 & 3    & 8.1 \\\hline
			\end{tabular}
    		\caption{改變行高並加入底色的表格}\label{basic_2}  %加入標題與標號參照的文字
		\end{table}
	
	\end{frame}
	
	\begin{frame}
		jkjijojojio\footnote{Hello World!}
	\end{frame}